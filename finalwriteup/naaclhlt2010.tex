%
% File naaclhlt2010.tex
%
% Contact: nasmith@cs.cmu.edu

\documentclass[11pt,letterpaper]{article}
\usepackage{naaclhlt2010}
\usepackage{times}
\usepackage{latexsym}
\usepackage{graphicx}
\usepackage{wrapfig}
\usepackage{url}
\usepackage{wrapfig}
\usepackage{color}
\usepackage{marvosym}
\usepackage{enumerate}
\usepackage{subfigure}
\usepackage{tikz}
\usepackage{amsmath}
\usepackage{amssymb}
\usepackage{hyperref}
\setlength\titlebox{6.5cm}    % Expanding the titlebox

\title{Personalized Web Search\\ Feature Analysis and Comparison of Ranking Algorithms}

\author{Biman Gujral\\
  {\tt bgujral1@jhu.edu}
  \And
  Rujuta Deshpande \\
  {\tt rdeshpa3@jhu.edu}}

\date{}

\begin{document}
\maketitle
\begin{abstract}
 We personalize web search rankings for a user by analyzing user search history from Yandex's search engine logs. We have in total 50 features, which we turn on and off for different runs and analyze the contribution of each feature to the resultant NDCG score. We compare the performance in terms of ranking accuracy using various Learning To Rank Algorithms - Random Forests, LambdaMART, Ranknet and AdaRank for a given set of features. 
\end{abstract}

\section{Introduction}
Learning to Rank is a technique that constructs ranking models for Information Retrieval Systems. The training data consists of a set of query-document pairs. Each query-document pair forms one data point and is associated with a feature vector. Every pair is assigned a relevance score. Documents associated with a query are ordered according to decreasing relevance scores. The task of a learning to rank model, is to assign a relevance score to each query-document pair in the test data and thereby, create a permutation over the different documents returned for a particular query which is similar to the ordering found in the train data. Personalized rankings, re-rank the returned results in an order that would be preferable to a given user. The preference is learnt from a user's past search history. It can also be learnt from his searches and clicks in the same sessions or in anterior sessions.
\section{Related Work}

\subsection{Algorithms}
Typically, learning to rank algorithms fall into three broad categories. They are pointwise, pairwise and listwise. 
\subsubsection{Pointwise Algorithms}
In pointwise algorithms, each query-document pair has a numerical score associated with it. Then, a learning-to-rank algorithm can transform into a regression problem of predicting a score given a query-document pair. If the scores take values from a finite set, this can even be a classification problem. Examples of pointwise algorithms are Random Forests. 
\subsubsection{Pairwise Algorithms}
In the pairwise approach, each pair of documents returned for a query is selected and between each member of a pair, one tries to determine the more relevant document. The goal is to minimize the inversions in a ranking.%%Reference
Example of pairwise algorithm is Ranknet.
\subsubsection{Listwise Algorithms}
Listwise algorithms try to either find the optimum score or minimize inversions, however, they average over all queries in the training data. LambdaMART and AdaRank are examples of listwise algorithms.
\subsection{Evaluation Metric}
There are several evaluation metrics used in learning-to-rank algorithms. We are using one called NDCG (Normalized Discounted Cumulative Gain).\\
The evaluation metric used is NDCG (Normalized Discounted Cumulative Gain). It is a metric between 0 and 1 that evaluates the ranking order. It is given by:

\begin{flalign*}
NDCG_k = \frac{DCG_k}{IDCG_k}
\end{flalign*}
where k denotes documents uptil rank k and DCG is Discounted Cumulative Gain, given by:
\begin{flalign*}
DCG_k = \sum_{i=1}^k{\frac{2^{rel_i} - 1}{log_2(i + 1)}}
\end{flalign*}
where rel is the actual relevance of the document provided by labels and i is the rank given by the algorithm. Thus, a highly relevant document ranked later in the list results in penalization and a low DCG. The Ideal DCG or IDCG gives the best ranking in accordance with the relevance values. Therefore, NDCG = 1 when the ideal ranking is obtained. 


\section{Our Approach}

We attempt at personalizing a user's ranking data based on the past history of his searches. This is a problem that Yandex - a Russian search engine had proposed on Kaggle. 
The key idea that the winning team used is creating a large number of features and then, they ran the Random Forests and LambdaMART algorithms to generate an NDCG score. 
We are creating a subset of features that they have used. However, we have provided a mechanism for creating a configuration file that turns features on and off, hence, allowing a user to test out how various feature configurations can affect the correctness of the ranking order. We also do a comparitive study of 4 algorithms - LambaMART, Random Forests, AdaRank and Ranknet using our different feature configurations. This allows us to determine how a pointwise algorithm would perform as compared to a list wise or pair wise algorithm on the same data set.\\
We experiment by varying different parameters used by each of these algorithms.

\subsection{Background}




\subsection{Data Preparation}
\label{sect:pdf}

For the production of the electronic manuscript you must use Adobe's
Portable Document Format (PDF). This format can be generated from
postscript files: on Unix systems, you can use {\tt ps2pdf} for this
purpose; under Microsoft Windows, you can use Adobe's Distiller, or
if you have cygwin installed, you can use {\tt dvipdf} or
{\tt ps2pdf}.  Note 
that some word processing programs generate PDF which may not include
all the necessary fonts (esp. tree diagrams, symbols). When you print
or create the PDF file, there is usually an option in your printer
setup to include none, all or just non-standard fonts.  Please make
sure that you select the option of including ALL the fonts.  {\em
  Before sending it, test your {\/\em PDF} by printing it from a
  computer different from the one where it was created}. Moreover,
some word processor may generate very large postscript/PDF files,
where each page is rendered as an image. Such images may reproduce
poorly.  In this case, try alternative ways to obtain the postscript
and/or PDF.  One way on some systems is to install a driver for a
postscript printer, send your document to the printer specifying
``Output to a file'', then convert the file to PDF.

For reasons of uniformity, Adobe's {\bf Times Roman} font should be
used. In \LaTeX2e{} this is accomplished by putting

\begin{quote}
\begin{verbatim}
\usepackage{times}
\usepackage{latexsym}
\end{verbatim}
\end{quote}
in the preamble.

Additionally, it is of utmost importance to specify the {\bf
  US-Letter format} (8.5in $\times$ 11in) when formatting the paper.
When working with {\tt dvips}, for instance, one should specify {\tt
  -t letter}.

Print-outs of the PDF file on US-Letter paper should be identical to the
hardcopy version.  If you cannot meet the above requirements about the
production of your electronic submission, please contact the
publication chairs above  as soon as possible.


\subsection{Feature Engineering}
\label{ssec:layout}

Format manuscripts two columns to a page, in the manner these
instructions are formatted. The exact dimensions for a page on US-letter
paper are:

\begin{itemize}
\item Left and right margins: 1in
\item Top margin:1in
\item Bottom margin: 1in
\item Column width: 3.15in
\item Column height: 9in
\item Gap between columns: 0.2in
\end{itemize}

\noindent Papers should not be submitted on any other paper size. Exceptionally,
authors for whom it is \emph{impossible} to format on US-Letter paper,
may format for \emph{A4} paper. In this case, they should keep the \emph{top}
and \emph{left} margins as given above, use the same column width,
height and gap, and modify the bottom and right margins as necessary.
Note that the text will no longer be centered.

\subsection{Libraries and Code}
\label{ssec:first}

Center the title, author's name(s) and affiliation(s) across both
columns. Do not use footnotes for affiliations.  Do not include the
paper ID number assigned during the submission process. 
Use the two-column format only when you begin the abstract.

{\bf Title}: Place the title centered at the top of the first page, in
a 15 point bold font.  (For a complete guide to font sizes and styles, see Table~\ref{font-table}.)
Long title should be typed on two lines without
a blank line intervening. Approximately, put the title at 1in from the
top of the page, followed by a blank line, then the author's names(s),
and the affiliation on the following line.  Do not use only initials
for given names (middle initials are allowed). Do not format surnames
in all capitals (e.g., ``Leacock,'' not ``LEACOCK'').  The affiliation should
contain the author's complete address, and if possible an electronic
mail address. Leave about 0.75in between the affiliation and the body
of the first page.
{\bf Abstract}: Type the abstract at the beginning of the first
column.  The width of the abstract text should be smaller than the
width of the columns for the text in the body of the paper by about
0.25in on each side.  Center the word {\bf Abstract} in a 12 point
bold font above the body of the abstract. The abstract should be a
concise summary of the general thesis and conclusions of the paper.
It should be no longer than 200 words.  The abstract text should be in 10 point font.

{\bf Text}: Begin typing the main body of the text immediately after
the abstract, observing the two-column format as shown in 
the present document.  Do not include page numbers.

{\bf Indent} when starting a new paragraph. For reasons of uniformity,
use Adobe's {\bf Times Roman} fonts, with 11 points for text and 
subsection headings, 12 points for section headings and 15 points for
the title.  If Times Roman is unavailable, use {\bf Computer Modern
  Roman} (\LaTeX2e{}'s default; see section \ref{sect:pdf} above).
Note that the latter is about 10\% less dense than Adobe's Times Roman
font.

\subsection{Problems Encountered}

{\bf Headings}: Type and label section and subsection headings in the
style shown on the present document.  Use numbered sections (Arabic
numerals) in order to facilitate cross references. Number subsections
with the section number and the subsection number separated by a dot,
in Arabic numerals. 

{\bf Citations}: Citations within the text appear
in parentheses as~\cite{Gusfield:97} or, if the author's name appears in
the text itself, as Gusfield~\shortcite{Gusfield:97}. 
Append lowercase letters to the year in cases of ambiguities.  
Treat double authors as in~\cite{Aho:72}, but write as 
in~\cite{Chandra:81} when more than two authors are involved. 
Collapse multiple citations as in~\cite{Gusfield:97,Aho:72}.

\textbf{References}: Gather the full set of references together under
the heading {\bf References}; place the section before any Appendices,
unless they contain references. Arrange the references alphabetically
by first author, rather than by order of occurrence in the text.
Provide as complete a citation as possible, using a consistent format,
such as the one for {\em Computational Linguistics\/} or the one in the 
{\em Publication Manual of the American 
Psychological Association\/}~\cite{APA:83}.  Use of full names for
authors rather than initials is preferred.  A list of abbreviations
for common computer science journals can be found in the ACM 
{\em Computing Reviews\/}~\cite{ACM:83}.

The \LaTeX{} and Bib\TeX{} style files provided roughly fit the
American Psychological Association format, allowing regular citations, 
short citations and multiple citations as described above.

{\bf Appendices}: Appendices, if any, directly follow the text and the
references (but see above).  Letter them in sequence and provide an
informative title: {\bf Appendix A. Title of Appendix}.

\textbf{Acknowledgment} sections should go as a last (unnumbered) section immediately
before the references.  

\subsection{Details of Tests}

{\bf Footnotes}: Put footnotes at the bottom of the page. They may
be numbered or referred to by asterisks or other
symbols.\footnote{This is how a footnote should appear.} Footnotes
should be separated from the text by a line.\footnote{Note the
line separating the footnotes from the text.}  Footnotes should be in 9 point font.

\subsection{Results}

{\bf Illustrations}: Place figures, tables, and photographs in the
paper near where they are first discussed, rather than at the end, if
possible.  Wide illustrations may run across both columns and should be placed at
the top of a page. Color illustrations are discouraged, unless you have verified that 
they will be understandable when printed in black ink.

\begin{table}
\begin{center}
\begin{tabular}{|l|rl|}
\hline \bf Type of Text & \bf Font Size & \bf Style \\ \hline
paper title & 15 pt & bold \\
author names & 12 pt & bold \\
author affiliation & 12 pt & \\
the word ``Abstract'' & 12 pt & bold \\
section titles & 12 pt & bold \\
document text & 11 pt  &\\
abstract text & 10 pt & \\
captions & 10 pt & \\
bibliography & 10 pt & \\
footnotes & 9 pt & \\
\hline
\end{tabular}
\end{center}
\caption{\label{font-table} Font guide. }
\end{table}

{\bf Captions}: Provide a caption for every illustration; number each one
sequentially in the form:  ``Figure 1. Caption of the Figure.'' ``Table 1.
Caption of the Table.''  Type the captions of the figures and 
tables below the body, using 10 point text.  

\section{Length of Submission}
\label{sec:length}

The NAACL HLT 2010 main conference accepts submissions of long papers
and short papers.  The maximum length of a long paper manuscript is
eight (8) pages of content and one (1) additional page of references
\emph{only} (appendices count against the eight pages, not the
additional one page).  The maximum length of a short paper manuscript
is four (4) pages including references.  For both long and short
papers, all illustrations, references, and appendices must be
accommodated within these page limits, observing the formatting
instructions given in the present document.  Papers that do not
conform to the specified length and formatting requirements are
subject to be rejected without review.

% Up to two (2) additional pages may be purchased from ACL at the
% price of \$250 per page; please contact the publication chairs above
% for more information about this option.

\section*{References}

Do not number the acknowledgment section.

\begin{thebibliography}{}

\bibitem[\protect\citename{Aho and Ullman}1972]{Aho:72}
Alfred~V. Aho and Jeffrey~D. Ullman.
\newblock 1972.
\newblock {\em The Theory of Parsing, Translation and Compiling}, volume~1.
\newblock Prentice-{Hall}, Englewood Cliffs, NJ.

\bibitem[\protect\citename{{American Psychological Association}}1983]{APA:83}
{American Psychological Association}.
\newblock 1983.
\newblock {\em Publications Manual}.
\newblock American Psychological Association, Washington, DC.

\bibitem[\protect\citename{{Association for Computing Machinery}}1983]{ACM:83}
{Association for Computing Machinery}.
\newblock 1983.
\newblock {\em Computing Reviews}, 24(11):503--512.

\bibitem[\protect\citename{Chandra \bgroup et al.\egroup }1981]{Chandra:81}
Ashok~K. Chandra, Dexter~C. Kozen, and Larry~J. Stockmeyer.
\newblock 1981.
\newblock Alternation.
\newblock {\em Journal of the Association for Computing Machinery},
  28(1):114--133.

\bibitem[\protect\citename{Gusfield}1997]{Gusfield:97}
Dan Gusfield.
\newblock 1997.
\newblock {\em Algorithms on Strings, Trees and Sequences}.
\newblock Cambridge University Press, Cambridge, UK.

\end{thebibliography}

\end{document}
